\documentclass{article}

\usepackage{header}

\usepackage[left=2cm,right=2cm, top=2cm,bottom=2cm,bindingoffset=0cm]{geometry}

\usepackage{graphicx}
% \graphicspath{.}

\usepackage{hyperref}  % so the reference URLs and citations are clickable
\usepackage{csquotes}  % needed for biblatex for babel
\usepackage[backend=biber]{biblatex}
\addbibresource{bibliography.bib}

\usepackage{titlepage}

\setUDK{004.9}
\setToResearch

\setTitle{Обратимые фильтры Блума и сравнение геномов}

% Выбрать одно из трех:
% КТ1 -- \setStageOne
% КТ2 -- \setStageTwo
% Финальная версия -- \setStageFinal
\setStageOne
%\setStageTwo
%\setStageFinal

\setGroup{209}
%сюда можно воткнуть картинку подписи
% \setStudentSgn{\smash{\includegraphics[scale=0.25]{my-sig.png}}}
\setStudentSgn{\includegraphics[scale=0.10]{my-sig.jpg}}
% \setStudentSgn{\smash{kjdfl}}
\setStudent{М.М.Марченко}
\setStudentDate{17.02.2022}
\setAdvisor{Григорий Аронович Кучеров}
\setAdvisorTitle{Ведущий научный сотрудник} %(научно-учебная лаборатория методов анализа больших данных)}
\setAdvisorAffiliation{CNRS, France}
\setAdvisorDate{}
\setGrade{}
%сюда можно воткнуть картинку подписи
\setAdvisorSgn{}
\setYear{2022}

\begin{document}

% Эта команда создает титульную страницу
\makeTitlePage

% Здесь будет автоматически генерироваться содержание документа
\tableofcontents

% Данное окружение оформляет аннотацию: краткое описание текста выделенным абзацем после заголовка
%\begin{abstract}
%    Проект в реализации теории,
%\end{abstract}

\section{Introduction}

Bloom Filter is probably one of the most recognizable probabilistic data
structures. It is included in many algorithmic courses and broadly used.


Фильтр Блума --- пожалуй, самая известная вероятностная структура данных. Фильтр Блума входит в программу 
большинства курсов по алгоритмам и, как известно, уже много где применен. Одна из вариаций фильтра Блума -
менее известный обратимый фильтр Блума (\textcite{GoMi2011}), занимает больше памяти, однако обладает сильно расширенной функциональностью.

Одним из преимуществ обратимого фильтра Блума является при наличии двух наборов данных A и B,
отличающихся на t, сжимать их до O(t) и восстанавливать (с помощью либо A, либо B) с низкой вероятностью ошибки.

Геном человека занимает большое количество памяти, однако, как известно, может отличаться лишь в 0.1\% мест.
Данная работа ставит целью оптимизировать использование памяти для хранения генома, а также поэксперементировать с сравнением фильтров с запакованными геномами и восстановлению разницы между ними. 

\section{Цели}
\begin{enumerate}
    \item Реализовать обратимый фильтр Блума на C++.
    \item Проинициализировать результат данными генома.
    \item Оптимизировать обратимый фильтр Блума по памяти и вероятности ошибки конкретно для случая хранения генома.
    \item В случае удовлетворительных результатов по оптимизации создать удобный интерфейс для запаковки и распаковки генома.
    \item Поэксперементировать с несколькими фильтрами, содержащими каждый свой геном.
    \item Попробовать восстановить разницу между двумя геномами запакованными в фильтры.
\end{enumerate}
\section{Фильтры Блума}
\subsection{Фильтр Блума}
Обычный фильтр Блума представляет собой битовый массив и набор из k хеш-функций.
По фильтру невозможно восстановить множество, хранящееся в нем, невозможно удалить элемент.
Поддерживает операции:
\begin{enumerate}
    \item insert(x) --- метод добавляет элемент x в множество, делая 1 биты соответствующие каждой из k хеш-функций от x. 
    \item query(x) --- отвечает на запрос есть ли элемент x в множестве. Метод проверяет каждый бит фильтра соответствующий каждой хеш-функции от x.
        Если хотя бы один из них равен 0 - элемента точно нет в множестве. Иначе дается положительный ответ (с маленькой вероятностью ложноположительный).
\end{enumerate}
\subsection{Обратимый фильтр Блума}
В отличие от обычного фильтра Блума, обратимый хранит пары из ключей и значений (в этом варианте ключи разные).
Обратимый фильтр Блума состоит из k хеш-функций и трех массивов: count (количество элементов в текущей ячейке),
keyxor (xor (или сумму) ключей, лежащих в данной ячейке) и valuexor (xor (или сумму) значений, лежащих в данной ячейке).
Обратимый фильтр Блума поддерживает операции:
\begin{enumerate}
    \item insert(key, value) --- метод вставки пары в множество. Для каждой хеш-функции от key к соответствующей ячейке count прибавляется 1, ячейка keyxor ксорится с key, ячейка valuexor c value.
    \item remove(key, value) --- метод удаления пары из множества. Для каждой хеш-функции от key от соответствующей ячейки count отнимается 1, ячейка keyxor ксорится с key, ячейка valuexor c value (поскольку xor ассоциативен, коммутативен и (a xor a = 0))
    \item get(key) --- метод получения значения по ключу. Если хотя бы одна из ячеек, соответствующих хеш-функциям от value, пустая (count = 0), то в множестве нет пары для key. Иначе, если есть ячейка с count = 1, то пара key содержится в соответствующей ячейке valuexor.
    \item list entries() --- метод, очищающий фильтр и выводящий все пары, содержащиеся в нем. Метод работает следующим образом: ищет по массиву count ячейки равные 1, выводит соответствующую пару и удаляет ее из фильтра. Операция повторяется пока не в фильтре не очистится (с маленькой вероятностью в непустом фильтре может остаться ни одной единицы - тогда вывод метода будет неполным).
\end{enumerate}

\subsection{Устойчивость к некорректной вставке и удалению}

Для случая, когда при использовании фильтра удаляют или вставляют несколько пар с одинаковым ключом и разными значениями в статье (\textcite{GoMi2011}) предлогается хранить дополнительный массив hashvaluesum, содержащий сумму хешей от значений, хранящихся в текущей ячейке.

\section{Сжатие генома в фильтр Блума}

Геномы хранятся как пары: название локуса (строка) и значение нуклеотида в локусе (число).
Геном состоит из 600 млн нуклеотидов, при этом геномы двух людей различаются в лишь около 3 млн мест.
Места различий могут быть разными для разных пар геномов, поэтому задача нахождения различий и задача сжатия генома не тривиальны.

\subsection{Сжатие и восстановление геномов}
Рассмотрим следующую задачу из статьи (\textcite{GoMi2011}):

Алиса (А) хочет отправить Бобу (Б) базу данных закодированную в виде набора из пар ключей и значений (при этом все ключи разные).
При этом у Б имеется старая версия данных и версии А и Б могут отличаться лишь на t (t значительно меньше размера всех данных).
А создает фильтр Блума размера порядка O(t), вставляет (метод insert) туда все пары из данных А и отправляет получившийся фильтр Б.
Б удаляет (метод remove) из полученного фильтра все пары из данных Б.
Теперь в фильтре находятся только пары отличающиеся у А и Б. У пар А значение в массиве count равно 1, у пар Б значение в массиве count равно -1.
Теперь Боб выводит (метод list entries) все оставшиеся пары.

Проблема со сжатием генома таким образом заключается в том, что ключи --- названия локусов в нашем случае, совпадают у А и у Б.
Поэтому при сжатии генома будет использована версия фильтра Блума устойчивая к нескольким парам с одинаковым ключом (то есть версия с дополнительным массивом G).


\subsection{Сравнение геномов} 
Известно, что при наличие двух фильтров Блума (обычных) для множеств A и B (соответственно $f_A$ и $f_B$), 
фильтром Блума для объединения A и B будет $f_A \vee f_B$, а фильтром пересечения будет $f_A \wedge f_B$.

При этом разность двух обратимых (но не устойчивых) фильтров Блума (count1 - count2, keyxor1 XOR keyxor2, valuesum1 - valuesum2)
будет содержать разность множеств А и B со знаком + и разность множеств B и A со знаком -.


\section{Текущие результаты}
На данный момент (17.02.2022) изучена литература (\textcite{GoMi2011}) по теме, а также на C++ реализован работающий обратимый фильтр Блума.

https://github.com/mmanchkin/bloom-lookup-table

(репозиторий приватный и нужно запросить доступ)


Эксперименты по оптимизации и сравнению можно будет начать, когда будет доступ к данным генома.

%\begin{enumerate}
%    \item Разработать алгоритм, восстанавливающий числа Бетти пространства по записи активности нейронов гиппокампа грызуна.
%    \item Обратная задача: разработать алгоритм, который, имея пространство известной топологии, генерирует
%        данные, достаточно похожие на запись нейронов гиппокампа грызуна, как если бы тот исследовал это пространство.
%\end{enumerate}
%Для решения этих задач планируется применить инструменты топологического анализа данных вкупе с
%результатами теории нейронных кодов.

%Для решения первой задачи взят за основу алгоритм,
%описанный в статье \textcite{CuIt08}, однако способ генерации активности не учитывает
%поведенческие особенности грызуна и использует упрощенную модель нейронов места, поэтому
%без модификаций этот алгоритм на реальных данных не работает, однако взят за основу.
% в случае компьютерной генерации записи активности нейронов существует алгоритм,
%  модель активности нейронов, использованная там, не учитывает

%Так как 

% Во введении надо кратко описать область, в которой будет ваша работа, потом  
% рассказать о поставленной задаче, далее о том, что вы будете делать.В конце  
% введения обычно принято писать обзор структуры содержательной части, чтобы   
% можно было сориентироваться в происходящем, не начиная читать содержательную 
% часть.

% \textcite{GGO} says
% Note also that very recently several constructions of~\cite{Elkik73} were clarified and simplified by Gabber and Ramero in~\cite[Chapter~5]{GabRam}. Some other source with url~\cite{GGO}

%\section{Методы}
%\subsection{Данные}
%Наши данные --- записи нейронной активности нескольких мышей, которые
%блуждают по круглым аренам с разной топологической структорой --- с 1, 2 или 3 пустотами.

%\subsection{Декодирование топологии}

%\subsection{Генерация нейронной активности}

%\section{Результаты}

%\section{Обсуждение}

% С этого момента глобальная нумерация идет буквами. Этот раздел я добавил лишь для демонстрации возможностей LaTeX, его можно и нужно удалить и он не нужен для курсового проекта непосредственно.
%\appendix

%\section{Математическая терминология}
% число, сохраняющееся после непрерывных преобразований пространства, 
%Число Бетти $\beta_i$ топологического пространства --- количество $i$-мерных
%пустот в нем. При этом нулевое число Бетти --- число компонент связности топологического пространства.
%Для графа $\beta_i$ --- количество независимых циклов в графе, например, у дерева
%$\beta_i = 0$. А, например, у геометрического объекта --- окружности $S_1$ --- $\beta = 1$.

\printbibliography

% Проведем небольшой обзор возможностей \LaTeX. Далее идет обзорный кусок, который надо будет вырезать. Он приведен лишь для демонстрации возможностей \LaTeX.
%
% \section{Нумеруемый заголовок}
% Текст раздела
% \subsection{Нумеруемый подзаголовок}
% Текст подраздела
% \subsubsection{Нумеруемый подподзаголовок}
% Текст подподраздела
%
% \section*{Не нумеруемый заголовок}
% Текст раздела
% \subsection*{Не нумеруемый подзаголовок}
% Текст подраздела
% \subsubsection*{Не нумеруемый подподзаголовок}
% Текст подподраздела
%
%
% \paragraph{Заголовок абзаца} Текст абзаца
% Формулы в тексте набирают так $x = e^{\pi i}\sqrt{\text{формула}}$. Выключенные не нумерованные формулы набираются либо так:
% \[
% x = e^{\pi i}\sqrt{\text{формула}}
% \]
% Либо так
% $$
% x = e^{\pi i}\sqrt{\text{формула}}
% $$
% Первый способ предпочтительнее при подаче статей в журналы AMS, потому рекомендую привыкать к нему.
%
% Выключенные нумерованные формулы:
% \begin{equation}
% \label{Equation1}
% % \label{имя-метки} эта команда ставит метку, на которую потом можно сослаться с помощью \ref{имя-метки}. Метки можно ставить на все объекты, у которых есть автоматические счетчики (номера разделов, подразделов, теорем, лемм, формул и т.д.
% x = e^{\pi i}\sqrt{\text{формула}}
% \end{equation}
% Или не нумерованная версия
% \begin{equation*}
% x = e^{\pi i}\sqrt{\text{формула}}
% \end{equation*}
%
% Уравнение \ref{Equation1} радостно занумеровано.
%
% Лесенка для длинных формул
% \begin{multline}
% x = e^{\pi i}\sqrt{\text{очень очень очень длинная формула}}=\\
% \tr A - \sin(\text{еще одна очень очень длинная формула})=\\
% \cos z \Im \varphi(\text{и последняя длинная при длинная формула})
% \end{multline}
%
% Многострочная формула с центровкой
% \begin{gather}
% x = e^{\pi i}\sqrt{\text{очень очень очень длинная формула}}=\\
% \tr A - \sin(\text{еще одна очень очень длинная формула})=\\
% \cos z \Im \varphi(\text{и последняя длинная при длинная формула})
% \end{gather}
%
% Многострочная формула с ручным выравниванием. Выравнивание идет по знаку $\&$, который на печать не выводится.
% \begin{align}
% x = &e^{\pi i}\sqrt{\text{очень очень очень длинная формула}}=\\
% &\tr A - \sin(\text{еще одна очень очень длинная формула})=\\
% &\cos z \Im \varphi(\text{и последняя длинная при длинная формула})
% \end{align}
%
% \begin{theorem}
% Текст теоремы
% \end{theorem}
% \begin{proof}
% В специальном окружении оформляется доказательство.
% \end{proof}
%
% \begin{theorem}[Имя теоремы]
% Текст теоремы
% \end{theorem}
% \begin{proof}[Доказательство нашей теоремы]
% В специальном окружении оформляется доказательство.
% \end{proof}
%
% \begin{definition}
% Текст определения
% \end{definition}
%
% \begin{remark}
% Текст замечания
% \end{remark}
%
% \paragraph{Перечни:} Нумерованные
% \begin{enumerate}
% \item Первый
% \item Второй
% \begin{enumerate}
% \item Вложенный первый
% \item Вложенный второй
% \end{enumerate}
% \end{enumerate}
%
% Не нумерованные
%
% \begin{itemize}
% \item Первый
% \item Второй
% \begin{itemize}
% \item Вложенный первый
% \item Вложенный второй
% \end{itemize}
% \end{itemize}
\end{document}
